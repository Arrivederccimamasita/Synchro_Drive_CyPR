\documentclass[a4paper,twoside]{article}

%% Language and font encodings
\usepackage[spanish]{babel}
\usepackage[utf8]{inputenc}
\usepackage[T1]{fontenc}


%% Sets page size and margins
\usepackage[a4paper,top=3cm,bottom=2cm,left=2.5cm,right=2.5cm,marginparwidth=0.5cm]{geometry}

\usepackage{amsmath}			%Paquete matemático
\usepackage{graphicx}
\usepackage[colorinlistoftodos]{todonotes}

\usepackage{hyperref}		%Paquete empleado para colocar hipervinculos
\hypersetup{
	colorlinks = true,
	linkcolor = black,
}

\usepackage{eurosym}
\usepackage{pdfpages}			%Sirve para incluir PDF en el documento
\usepackage{anysize}			%Podremos colocar imagenes de cualquier tamaño
\usepackage{subfig}				%Nos permitira colocar varias imagenes en una figura
\usepackage{float}				%Podremos crear y colocar boxes donee queramos
\usepackage[export]{adjustbox}

%Colocamos cabeceras y pies de pagina
%(CONSULTA: http://edicionesoniricas.com/maquetar-latex-encabezados-pies-pagina/)
%(CONSULTA2: https://es.sharelatex.com/learn/Headers_and_footers)
%\bfseries es análogo a \textbf{}
% \leftmark-> Adds name and number of the current top-level structure (section for article) in uppercase letters.
%\rightmark-> Adds name and number of the current next to top-level structure (subsection for article) in uppercase letters.
\usepackage{fancyhdr}		%Paquetes necesarios
\pagestyle{fancy}			%Borra los parametros por defecto
\fancyhf{}
\fancyhead[RO,LE]{\bfseries\thepage}
\fancyhead[LO,RE]{\bfseries\rightmark}
%Nos aseguramos de que en las paginas plain, no haya ni cabeceras ni lineas
\fancypagestyle{plain}
{
	\fancyhead{} % elimina cabeceras en paginas "plain"
	\renewcommand{\headrulewidth}{0pt} % así como la raya
}

%Definimos las lineas divisoras de las cabeceras y pie de pagina
\renewcommand{\headrulewidth}{1pt}	%Define el grosor de la línea de head
\renewcommand{\footrulewidth}{0pt}		%Define el grosor de la linea foot (Si no queremos linea, 0pt)
\addtolength{\headheight}{0.5pt} % espacio para la raya

%Librerias para introducir código de Matlab
%\usepackage{bigfoot} % to allow verbatim in footnote
\usepackage[numbered,framed]{matlab-prettifier}

\lstset{
	style              = Matlab-editor,
	basicstyle         = \mlttfamily,
	escapechar         = ",
	mlshowsectionrules = true,
}

% Pie de pagina
%\fancyfoot{} % limpia el pie
\fancyfoot[C]{- \thepage -} % número de página centrado

%Nos generará texto para pruebas de maquetado
\usepackage{lipsum}

% Se varia el limite de colimnas de latex
\setcounter{MaxMatrixCols}{11}
\usepackage{lscape}
%----------------------------------------------------------------------------------------------------------------------------------
\begin{document}
\begin{titlepage}
	\centering
\Huge{\textbf{CONTROL Y PROGRAMACIÓN DE ROBOTS}} \\
\Huge{\textit{Proyecto de robotica movil}}\\

\vspace{1cm}
\LARGE{Grado en Ingeniería Electrónica, Mecatrónica y Robótica}\\
\rule{\textwidth}{0.1mm}
%  %%%%% Este trozo de codigo es para insertar imagenes %%%%%%%
%\begin{figure}[h!]
%	\centering
%	\includegraphics[width=1\textwidth]{brazo_portada}
	%\caption{textodelaleyenda}
%\end{figure}
% %%%%%%%%%%%%%%%%%%%%%%%%%%%%%%%%%%%%%%%%%%%%%%%%%%%%%%%%%%%%%
\vspace{3cm}
\rule{\textwidth}{0.1mm}
\Large{\textbf{Autores:} Montes Grova, Marco Antonio\\
 Lozano Romero, Daniel\\
 Mérida Floriano, Javier}
\end{titlepage}
\tableofcontents
\newpage
% %%%%%%%%%%%   INTRODUCCION %%%%%%%%%%%%%%%%%%
\section{Introducción al proyecto}
En el proyecto que sigue a continuación se desarrollará el modeloado y control de un robot movil tipo síncrono, la principal caracteristica descable de éste tipo de robots recae en el mecanismo mecánico interno que posee mediante el cuál se podrán mover 3 ruedas empleando únicamente 2 motores.\\
Con uno de los motores se desplazará en línea recta y con el otro se le dará el ángulo de giro deseado sobre sí mismo.\\
% SEGUIR HABLANDO DE MOVIDAS TEORICAS UN PSEUDO LARGO TRECHO ETC ETC ETC ETC



\section{Análisis cinemático}
	\subsection{Obtención del modelo cinemático directo y su jacobiano}
	\subsection{Obtención del modelo cinemático inverso}
	\subsection{Definición de las trayectorias de lazo abierto del robot}
		\subsubsection{Verificación de la actuación senoidal}

\section{Control dinámico}
	\subsection{Implementación de diversos algoritmos de control}
		\subsubsection{Control a un punto}
		\subsubsection{Control a una linea}
		\subsubsection{Control a una trayectoria}
		\subsubsection{Control a una postura}
	\subsection{Ley de control \textit{Persecución pura}}

\section{Anexos y conclusiones}
\end{document}
